\documentclass[
a4paper
,11pt
,landscape
,russian
,twocolumn
]{letter}

\usepackage{fontspec}
\usepackage{xunicode}
\usepackage{xltxtra}
\usepackage{polyglossia}
\setromanfont[Mapping=tex-text]{Liberation Serif}
\setsansfont[Mapping=tex-text]{Liberation Sans}
\setmonofont[Mapping=tex-text]{Liberation Mono}
\setdefaultlanguage{russian} % Word wrap.
\usepackage[margin=0.5in]{geometry}
\usepackage{tabto}

\newcommand\vsp{\vspace{.2in}}
\newcommand\tto{\tabto{2in}}

\begin{document}
\thispagestyle{empty}

\begin{Large}
Биполярный транзистор
\end{Large}

Обозначения и выводы заимствованы из книги У. Титце, К. Шенк
<<Полупроводниковая схемотехника>>, М. <<Мир>>, 1982.

$U_{BE}$~---~напряжение база-эмиттер, или падение напряжения на
эмиттерном переходе незапертого транзистора. $\approx 0.6V$ для
кремниевых транзисторов, $\approx 0.2V$ для германиевых. При расчетах
значения можно считать постоянными, независимо от режима/коллекторного
тока (экспоненциальная стена).

$U_{CE}$~---~напряжение коллектор-эмиттер.

$U_T=\frac{kT}{e_0}\approx 25.5mV$~---~термический потенциал при
комнатной температуре.

$S=\frac{I_C}{U_T}$~---~крутизна характеристики: прирост коллекторного
тока $I_C$ в зависимости от прироста напряжения база-эмиттер $U_{BE}$.

$\beta$~---~коэффициент усиления по току: прирост
коллекторного тока в зависимости от прироста тока базы.

$r_{BE}=\frac{\beta}{S}=\frac{\beta U_T}{I_C}$~---~входное сопротивление
(дифференциальное сопротивление база-эмиттер).

$U_Y$($U_A$ в последующих изданиях)~---~напряжение Эрли; определяет
выходное (дифференциальное) сопротивление $r_{CE}$. Значение является
свойством транзистора и находится в пределах 40\,--\,200$V$.

$r_{CE}=\frac{U_Y}{I_C}$~---~выходное сопротивление.

$A=-Sr_{CE}=-\frac{I_C}{U_T}\frac{U_Y}{I_C}=-\frac{U_Y}{U_T}
\approx-\frac{100}{0.025}=-4000$~---~максимально возможный
коэффициент усиления по напряжению для схемы с общим эмиттером.
Знак <<минус>> означает поворот фазы на $180^{\circ}$.

\vsp

При постоянной температуре приросту напряжения $U_{BE}$ на $60mV$
соответствует десятикратный прирост коллекторного тока $I_C$;
температурная зависимость при нагреве составляет
$\approx -1.7\frac{mV}{K}$
(при неизменном токе коллектора $I_C$ $U_{BE}$ падает на
$1.7mV$ на градус). Таким образом, фиксируя напряжение на
эмиттерном переходе, задать желаемый постоянный ток $I_C$
невозможно: при нагреве транзистора за счет коллекторного
тока, последний будет неуправляемо расти. Все практические
схемы обеспечивают температурную стабильность за счет
отрицательной обратной связи.

В первом приближении можно сказать, что усилитель описывается
входным сопротивлением $r_e$, выходным сопротивлением $r_a$, и
коэффициентом усиления по напряжению $A$.

\vsp

\begin{Large}
Усилитель на одном транзисторе
\end{Large}

\vsp

$R_C$, $R_E$~---~сопротивление нагрузки, в цепи коллектора и
эмиттера соответственно. Дальнейшие равенства справедливы для
случая, когда $r_{CE}=\frac{U_Y}{I_C}$ много больше сопротивления
нагрузки, что справедливо в большинстве практических случаев.

Символ $\|$ означает параллельное соединение сопротивлений.

\vsp

\begin{raggedright}

\textit{Схема с общим эмиттером}:

$r_e=r_{BE}=\frac{\beta}{S}=\frac{\beta U_T}{I_C}$
\tto
{\small сопротивление умеренное; можно повысить}
\linebreak
\tto
{\small только за счет применения транзистора}
\linebreak
\tto
{\small с высоким значением} $\beta$

$r_a=R_C\|r_{CE}\approx R_C$
\hspace{.2in}
$A=-S R_C=-\frac{I_C R_C}{U_T}$

\vsp

\textit{Схема с общей базой}:

$r_e=\frac{1}{S}$
\tto
{\small сопротивление весьма низкое}

$r_a\approx R_C$
\hspace{.2in}
$A=S R_C$

\vsp

\textit{Схема с общим коллектором (эмиттерный повторитель)}:

$r_e\approx\beta R_E$
\tto
{\small сопротивление весьма высокое}

$r_a\approx\frac{1}{S}$
\tto
{\small сопротивление весьма низкое}

$A\approx 1$

\end{raggedright}

\end{document}
