\documentclass[
a4paper
,11pt
,landscape
,russian
,twocolumn
]{letter}

\usepackage{fontspec}
\usepackage{xunicode}
\usepackage{xltxtra}
\usepackage{polyglossia}
\setromanfont[Mapping=tex-text]{Liberation Serif}
\setsansfont[Mapping=tex-text]{Liberation Sans}
\setmonofont[Mapping=tex-text]{Liberation Mono}
\setdefaultlanguage{russian} % Word wrap.
\usepackage{lipsum}
\usepackage[margin=0.5in]{geometry}

\begin{document}
\thispagestyle{empty}

\begin{Large}
Биполярный транзистор
\end{Large}

\vspace{.2in}

$U_{BE}$~---~напряжение база-эмиттер, или падение напряжения на
эмиттерном переходе незапертого транзистора. $\approx 0.6V$ для
кремниевых транзисторов, $\approx 0.2V$ для германиевых.

$U_{CE}$~---~напряжение коллектор-эмиттер.

$U_T=\frac{kT}{e_0}\approx 25.5mV$~---~термический потенциал при
комнатной температуре.

$S=\frac{I_C}{U_T}$~---~крутизна характеристики: прирост коллекторного
тока $I_C$ в зависимости от прироста напряжения база-эмиттер $U_{BE}$.

$\beta$~---~коэффициент усиления по току: прирост
коллекторного тока в зависимости от прироста тока базы.

$r_{BE}=\frac{\beta}{S}=\frac{\beta U_T}{I_C}$~---~входное сопротивление
(дифференциальное сопротивление база-эмиттер).

$U_Y(U_A)$~---~напряжение Эрли; определяет выходное (дифференциальное)
сопротивление $r_{CE}$. Значение является свойством транзистора и
находится в пределах 40--200V.

$r_{CE}=\frac{U_Y}{I_C}$~---~выходное сопротивление.

$A=Sr_{CE}=\frac{I_C}{U_T}\frac{U_Y}{I_C}=\frac{U_Y}{U_T}
\approx\frac{100}{0.025}=4000$~---~максимально возможный
коэффициент усиления по напряжению для схемы с общим эмиттером.

\vspace{.2in}

При постоянной температуре приросту напряжения $U_{BE}$ на $60mV$
соотвествует десятикратный прирост коллекторного тока $I_C$;
температурная зависимость составляет $\approx -1.7\frac{mV}{K}$
(при неизменном токе коллектора $I_C$ $U_{BE}$ падает на
$1.7mV$ на градус). Таким образом, фиксируя напряжение на
эмиттерном переходе, задать желаемый постоянный ток $I_C$
невозможно: при нагреве транзистора за счет коллекторного
тока, последний будет неуправляемо увеличиваться. Все практические
схемы обеспечивают температурную стабильность за счет
отрицательной обратной связи.

В первом приближении можно сказать, что усилитель описывается
входным сопротивлением, выходным сопротивлением, и коэффициентом
усиления по напряжению.

\vspace{.2in}

\begin{Large}
Усилитель на одном транзисторе
\end{Large}

\vspace{.2in}

Схема с общим эмиттером:

Схема с общей базой:

Схема с общим коллектором (эмиттерный повторитель):

\end{document}
