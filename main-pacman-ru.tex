\documentclass[
a4paper
,11pt
,landscape
,russian
,twocolumn
]{letter}

\usepackage{fontspec}
\usepackage{xunicode}
\usepackage{xltxtra}
\usepackage{polyglossia}
\setromanfont[Mapping=tex-text]{Liberation Serif}
\setsansfont[Mapping=tex-text]{Liberation Sans}
\setmonofont[Mapping=tex-text]{Liberation Mono}
\setdefaultlanguage{russian} % Word wrap.
\usepackage[margin=0.5in]{geometry}
\usepackage{tabto}

\newcommand\hsp{\hspace{.2in}}
\newcommand\vsp{\vspace{.2in}}
\newcommand\tto{\tabto{2in}}
\usepackage{hyperref}

\begin{document}
\thispagestyle{empty}

\begin{Large}
	pacman \& yay, самое необходимое
\end{Large}

\vsp

\texttt{pacman} является единственным официальным менеджером пакетов для
\texttt{Manjaro} и \texttt{Archlinux}; \texttt{yay}~---~одним из многих
неофициальных.
К опциям \texttt{pacman} \texttt{yay} добавляет набор своих, и обеспечивает
удобную работу с AUR (Arch User Repository; неофициальная коллекция пакетов,
собираемых, как правило, из исходных текстов на компьютере пользователя).

\texttt{\$ yay -Sy}\hsp обновить индексы репозиториев, прописанных в
\textit{/etc/pacman.conf}.

\texttt{\$ yay -Su}\hsp обновить все пакеты, новые версии которых
содержатся в индексах.

\texttt{\$ yay -Syu}\hsp обновить индексы, а затем и пакеты.

\texttt{\$ yay -Yc}\hsp удалить ненужные зависимости. Применять
с осторожностью.

\texttt{\$ [yes |] yay -Sc}\hsp очистить кэш пакетов, оставив только
текущие версии. Утилита \texttt{yes} ответит ``\textit{yes}'' на вопросы
менеджера пакетов.

\texttt{\$ yay -Scc}\hsp очистить кэш пакетов полностью.

\vsp

\begin{Large}
	установка и удаление пакетов
\end{Large}

\vsp

\texttt{\$ yay -Ss geany}\hsp искать пакеты, имеющие отношение к
\texttt{geany}.

\texttt{\$ yay -S geany-plugins}\hsp установить желаемый пакет, в том
числе из AUR.

\texttt{\$ yay -U <local file>}\hsp установить пакет не из репозиториев,
например, собранный вручную.

\texttt{\$ yay -R geany}\hsp удалить пакет.

\texttt{\$ yay -Rsn geany}\hsp удалить пакет со всеми зависимостями.
Применять с осторожностью.

\vsp

\begin{Large}
	исследование пакетов
\end{Large}

\vsp

\texttt{\$ yay -Qi}\hsp информация о пакете и его зависимостях.

\texttt{\$ yay -Ql}\hsp файлы пакета.

\texttt{\$ yay -Fy}\hsp обновить базу содержимого пакетов.

\texttt{\$ yay -F latexmk}\hsp найти пакет, содержащий \texttt{latexmk}.

\vsp

\begin{Large}
	статистика
\end{Large}

\vsp

\texttt{\$ exstat}\hsp место, занимаемое пакетами в системе
(утилита определена в \textit{\~~/.zshrc}).

\texttt{\$ exst}\hsp то же, что и \texttt{exstat}; только десять
самых тяжелых пакетов.

\vsp

Документация
\newline
по-русски:~~\url{https://wiki.archlinux.org/title/Pacman_(%D0%A0%D1%83%D1%81%D1%81%D0%BA%D0%B8%D0%B9)}

\end{document}
