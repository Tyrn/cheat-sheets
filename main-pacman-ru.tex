\documentclass[
a4paper
,11pt
,landscape
,russian
,twocolumn
]{letter}

\usepackage{fontspec}
\usepackage{xunicode}
\usepackage{xltxtra}
\usepackage{polyglossia}
\setromanfont[Mapping=tex-text]{Liberation Serif}
\setsansfont[Mapping=tex-text]{Liberation Sans}
\setmonofont[Mapping=tex-text]{Liberation Mono}
\setdefaultlanguage{russian} % Word wrap.
\usepackage[margin=0.5in]{geometry}
\usepackage{tabto}

\newcommand\hsp{\hspace{.2in}}
\newcommand\vsp{\vspace{.2in}}
\newcommand\tto{\tabto{2in}}

\begin{document}
\thispagestyle{empty}

\begin{Large}
    pacman \& yay
\end{Large}

\texttt{pacman} является единственным официальным менеджером пакетов для
\texttt{Manjaro} и \texttt{Archlinux}; \texttt{yay}~---~одним из многих
неофициальных.
К опциям \texttt{pacman} \texttt{yay} добавляет набор своих, и обеспечивает
удобную работу с AUR (Arch User Repository; неофициальная коллекция пакетов,
собираемых, как правило, из исходных текстов на компьютере пользователя).

\texttt{\$ yay -Sy}\hsp обновить индексы репозиториев, прописанных в
\textit{/etc/pacman.conf}.

\texttt{\$ yay -Su}\hsp обновить все пакеты, новые версии которых
содержатся в индексах.

\texttt{\$ yay -Syu}\hsp обновить индексы, а затем и пакеты.

\vsp

\begin{Large}
Усилитель на одном транзисторе
\end{Large}

\vsp

$R_C$, $R_E$~---~сопротивление нагрузки, в цепи коллектора и
эмиттера соответственно. Дальнейшие равенства справедливы для
случая, когда $r_{CE}=\frac{U_Y}{I_C}$ много больше сопротивления
нагрузки, что справедливо в большинстве практических случаев.

\end{document}
